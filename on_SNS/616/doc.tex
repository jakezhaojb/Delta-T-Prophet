\documentclass{article}
\usepackage{xeCJK}
\usepackage{color}
\usepackage{hyperref}
\setCJKmainfont{AR PL UKai CN}
\definecolor{colorname}{rgb}{0,0.12,0.60}
\begin{document}
\moveleft.5\hoffset\centerline{\huge\bf EE/CS想出国的童鞋可以来看一看 }
\moveleft.5\hoffset\centerline{\large by Delta-T}
\vspace{10pt}

终于有机会把想说的话说出来,把想做的事付诸于行动,希望可以帮助到更多的人。\\

首先,我言简意赅。我是赵俊博,来自武汉大学电子信息学院,现在大四,马上赴美学习。今天写这个东西,不是想黑留学中介或是怎样,我只是想跟那些现在大二或大三的EE/CS的学弟学妹们小说两句申请这件小事儿,还有我和两个朋友的一些想法。\\

 
如果你有兴趣?请读下去。
\section{MS申请究竟是怎么回事?}
MS或者PhD,和本科申请差别很大。从两个词来说,Scale and pure. 具体来说呢,Scale是什么?规模。不管怎么说MS的申请难度比本科大上几倍的终极原因就是这个scale,也就是说你的竞争者会比本科申请者大上倍;pure——纯粹,这是我最想说的问题,也是很多同学的误区。MS的申请是建立在4年本科专业学习的基础上的,所以你的PS里面最好少写一点学生会,少写一点打篮球,你该表现出来的是你的专业实力,下文简称“干货”。在拼干货的战场上,写那些有的没的真心是没用的。

\section{留学中介在做什么?}
留学中介,很多人对它的两句话形容就是——签之前感觉深不可测可以创造奇迹,申完之后觉得狗屁不是。就是说你在签之前,对出国一般都是一无所知,而你去签约也纯粹就是个心理安慰。但是我们不能一概而论,下文起,我将使用“好中介”和“坏中介”来诠释留学中介的角色,并且我需要声明的“好中介”和“坏中介”的相关事迹都是真实的,都是我根据自己的经历和身边朋友的中介经历来描述的。OK,那么他们能做什么呢?\\
 
(1) 帮你选校——通常情况下他们只是根据他们以往的记录。好中介会有个系统,通过检索引擎找到之前申请的案例;坏中介通常不想花钱做个系统,于是就有很多本子,他们就在那里翻,翻到谁是谁。轮到我说话了,首先选校这个事远远比想象的更加复杂——一句话概括就是你申请实际上对的是program,不是department,不是school,不是university。也就是说你申请的时候同一个学校甚至都是CS的不同program申请的难度可能会大相径庭(可以用大名鼎鼎的USC)来作为例子。而好坏中介通常的做法,就是根据申请到的学校,或者专业(比如说CS),来用你的水平和记录中的前辈做对比,之后做出论断——你行还是不行。这里面的问题就太大了,首先对比的依据基本就是T和G,至于干货的东西中介的做法就是忽略不计。T和G究竟是什么东西,待我慢慢说。\\

(2) 帮你写文书、推荐信和简历——这里面的问题就更多了。第一,推荐信。好坏中介的区别就是坏中介写出来的句子狗屁不通(真实的,我真的没有夸张);好中介基本还是负责的,不会乱写。那么即使如此,我想说推荐信这个东西是教授写的,你觉得让一帮中介的文书老师可能写出教授的高度吗?第二,文书——我并不是夸张,中介的做法就是和本科申请一个套路,本科和MS申请的区别我提过了。PS是展现你专业实力的最好机会,中介通常不会给你写专业的东西,原因太简单了——你的干货他们不懂。申请CS的同学们你们让连hello world都写不出来的人给你写文书吗?申请EE的同学你们要让一个连Fourier Transform都不懂的人给你写文书吗?申请CE的同学你们可以问问你的中介他们知不知道什么总线。最后,我想说一句,作为EE/CS/CE的同学,你们要是用微软的office来做你的文字材料来做你的申请你就输了一半了,下面会告诉你们一个叫TeX的东西。\\

(3) 寄送材料——好中介这一点做的肯定是没问题的,坏中介就是什么都不查,一通乱寄。问题是,每个学校对材料的要求是不同的,比如说有的要证明类材料有的不要,有的要两份成绩单有的不要,而且学校的地址也是需要自己查的。同学们,我想说的就是中介真的不会认真对待这事,尤其是坏中介,他们通常就是“我们每年都是这么寄的,你就这么来吧!”。呵呵,要是人家办公室搬家了呢?至少我每个学校的寄送都是我自己查的,中介不会帮你干这个事。DHL的话一份200,Fedex更贵……同学们,你觉得他们靠谱么。。。?\\

(4) 填写网申系统——这是个很烦的差事,两点要说:第一,你让中介填,放心吗?反正我是不放心,都是我自己填的,之后让我的中介(其实是负责的好中介,但我依然不放心)帮忙review。怀中介的做法就是一通狂写,不管正误;第二,童鞋你要是连网申系统都填不了的话,你就可以不用去美国了。。这真的是最easy的东西了,你要是以后看看租房的合同,SSN的申请乱七八糟的你不是要抓狂了?所以我想说别放弃这个机会,看看美国的学校是怎么录人的。\\
 
要说的就差不多,我想说我是一个签了中介的人,我的中介老师很负责,但是他们能做的真的很有限,因为我说过了,干货至上。至于坏中介我也见识过,除了呵呵二字我别无它词。\\

\section{我们是谁?}
我们是Delta-T,首先来介绍一下我们团队的人吧:
\begin{itemize}
\item{李彦迪,斯坦福大学EECS研究生,Oracle中国系统研究院实习生。关于李大神,这两条就够了,不需多说。}
\item{陈硕烁,弗吉尼亚大学Computer Engineering研究生,全奖RA,(如果你查查每年弗吉尼亚大学的招生量你就明白陈总这个RA有多牛逼了)Xilinx实习生,工程能力爆表。他主要负责指导申请的部分是:网络、嵌入式、体系结构。}
\item{赵俊博,纽约大学Data Science研究生,追随深度学习大牛而去、放弃了CMU等名校,豆瓣实习生。主要负责计算机视觉、机器学习和信号处理的申请指导。}
\end{itemize}
BTW,目测会有其它更优秀的朋友也会很快加入我们!(在线简历、联系方式请见最后。)

\section{我们想做什么?}
说了这么多,我们团队要做的事情,也是我们一直想做的事情。走了一圈申请流程,我们都给美国的老师、小蜜打过电话,邮件来往无数次,我们相信我们对申请这个事有了比留学中介更加深刻的认识。以我们学校为例,珞珈山下人才攒动,有能力、有本事还有志向的人有很多,但是很多人不懂出国是什么,然后被别人忽悠得团团转,我们想做的就是这个——以专业的眼光,加上申请的经验告诉他们:要出国你要怎么做?要出国,干货怎么积累?要出国,语言考试怎么办?我们想带来的是全新的申请理念、或者是“干货出国路”。\\

具体地:
\begin{itemize}
\item{选校,哦,应该说是“选项目”——这没那么难,但是需要查。每个学校的项目设置不同,方向不同,还有很重要的是选课的灵活性不同——这一点可能你不理解,简单的说就是假设你去学的EE,但是你对一个叫做电磁场的东西怕的要死。那么有的学校就会强制你上这门课,有的学校就放宽它。这里面的门道儿,可多着呢~}

\item{材料制作。两个问题——1. TeX,还是那句话,我们不建议申请EE/CS的同学使用word。就我来看,美国所有的研究生、PhD或是教授个人主页上的简历都是使用TeX排版而成。TeX是专业的排版工具,如果你是做技术的,用“所见即所得”的word显得不是很专业~2. 专业视角,写PS我们不想写那么多的学生会之类,我们会告诉你怎么展现自己的技术能力,学习能力和科研能力,即使在水平有限的情况下也可以展现给美国人看。}

\item{干货辅导。我们3个人应该说是同龄人里面,各个方向上比较专业的人了。我们团队会根据你的方向做出一些相应的”干货辅导“。你觉得你的简历上出现IEEE publications是不可能的事吗(我们是以后要混学术圈的人,不干买卖这种肮脏的事)?只要你肯努力,真正得完成一点工作,加上我们的帮助和指导,Impossible is nothing.}

\item{我们的名字叫做Delta-T,意思就是我们会成为“时差党”。也就是在美国的学生,在去年一年的时间里面积淀的专业申请经验,加上我们现在可以更加贴近于美国大学的招生办,更加深刻而明确得体会他们的小蜜是如何工作的、他们的教授是怎么看人的。}
\end{itemize}

\section{续上一章}
当然,我们不是NGO,上述事情是要收费的,但是也不是黑心中介。如果收费目前没有定论,但是可以承诺的是我们的费用会比你找中介的费用低N倍,收费完全是为了获得与我们劳动等值的东西,不为别的。如果你不足够信任我们,欢迎你来找我们聊聊。\\
 
写这篇文章的时候,我感觉只有真正走上教育相关的事业的时候才能明白其中的高尚之处——这也是我们团队为什么早就准备做这事情的原因。传道、授业、解惑者称为老师,我们虽然资历尚浅,但是很想肩负起这样的责任,以帮助那些更多的有本事、潜质的同学们。我想说,我依然想用洋务运动的那句话来鼓舞大家所有人:师夷长技以制夷。美帝没有我们想的那么伟岸和高不可攀,我们不要妄自菲薄;我们要充满自信。

\section{我还想和看到这篇文章的家长唠叨几句}
\begin{itemize}
\item{我申请过,也找了中介,我的中介是个好中介,老师很负责任,但是即便如此,我感觉很多核心的事情也需要我自己来做。我现在很理解做爸妈的心情,就像我理解我爸妈去年为什么那么坚持给我找留学中介一样——为了安心,能出国。其实我想说的是,出国这件事情本该是很纯粹的事情。它就像相亲大会,您带着您的孩子到一个平台上,自信得对他/她进行一番介绍,有很多个学校不同的“姑娘”站在您和您的孩子面前。你们面面相觑,侃侃而谈,倘若两情相悦便握手言和,倘若话不投机便一拍两散,当然啦,“姑娘”们通常会高贵冷艳一点。其实说白了就是这么简单——只有让您的孩子展现出一种很“专业”且真实的能力才能让对方净折腰。我素来反对市场上中介所谓的“包装”,甚至于造假。他们那些所谓的包装,不需要老美亲自出马,以我们的专业视角一看就呵呵了……}
\item{“排名保证”这个东西绝对百分之百是一个毒药。我觉得这是一个很扯的事情,很多中介(以坏中介居多)喜欢说“我保证您的孩子可以申请到前50的学校”;更有甚者根据您孩子最终的申请结果来定您要缴纳多少费用。在此我说三点——第一,对于MS或者PhD,综排这个东西几乎没有什么意义。我承认总是能听到很多人说“上了名校就可以找到好工作”。我想说Ok,金融行业有可能是这样,我不懂。但是对于技术行业,更多得在专排。因此这种一边倒的“保证”是没有意思的事情;第二,不是排名越高的就越难申请,这是个广泛的误区,反之毅然。比如说USC,排名很高,学校也很好,但是招生量很大,很多人甚至把它当成“保底”校。又比如说大概您们没听说过的一个叫UMASS的学校,在计算机视觉和人工智能上强到不行,发布了著名的LFW人脸数据库,综排98的学校,毅然把我拒掉。第三,研究生的话很多学校有两种说法,即MEng和MS。顾名思义,前者有的会很好申,比如说著名藤校Cornell大学,后者却总是难得不行。把这个因素加到所谓的“保证”里面,您不觉得很可笑吗?}
\item{商业靠的是一种“信息上的不对称”——说白了就是您不懂,商家就狠坑你。我认为这是大多数留学中介的做法,但是在这个问题上您不妨试试我们。}
\item{Delta-T是个新的团队,我们势单力薄,确实比不上市场上那些大型中介。但是我们非常有信心做到比他们更专业更靠谱。我说过我们不是为了钱做这些事,而是很想为很多迷茫的学弟学妹做点什么,其实我想说他们能做到的远远比现在更多!我们团队目前由我们三个“巨能折腾”的有志青年组成,以后可能还有人加入。我们日后也会是混学术圈的人,在此表示我们会以学术界的诚恳、审慎和责任感去做我们应该做的事。还是那句话,您可以联系我们。}
\end{itemize}

\section{联系方式}
\begin{itemize}
\item{如果你对网络、嵌入式、体系结构感兴趣,请联系:陈硕烁\\
在线简历:0x000fff.github.io\\
Skype ID: shawnchen.ss\\
Email: sschenshawn@gmail.com\\
QQ: 2339815524}
\item{如果你对计算机视觉、机器学习、信号处理感兴趣,请联系:赵俊博\\
在线简历1:zhaojunbo.github.io\\
在线简历2:\href{https://www.dropbox.com/s/layknk0qbke541c/res9a_JunboZhao.pdf}{\textcolor{colorname}{On Dropbox}}\\
Skype ID: zhaojunbo1992\\
Email: zhaojunbo1992chasing@gmail.com\\
QQ: 309390493}
\item{其他的方向或咨询,请联系:李彦迪\\
Email: yandi@stanford.edu\\
QQ: 1609826194}
\end{itemize}
\\
\vspace{20pt}
{\large\bf Delta-T欢迎大家骚扰~}

\end{document}
 

