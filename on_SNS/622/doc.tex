\documentclass{article}
\usepackage{xeCJK}
\usepackage{color}
\definecolor{colorname}{rgb}{0,0.12,0.60}
\definecolor{colorname2}{rgb}{0.60,0.12,0}
\usepackage{hyperref}
\setCJKmainfont{AR PL UKai CN}
\begin{document}
\moveleft.5\hoffset\centerline{\huge\bf Delta-T开始了! }
\vspace{5pt}
\moveleft.5\hoffset\centerline{\large\bf 申请EE/CS的同学看过来 }
\vspace{15pt}
%\moveleft\hoffset\vbox{\hrule width\resumewidth height 1pt}\smallskip
%\moveleft.5\hoffset\centerline{\large 参与人: Barry, Jake and Shawn}
这一篇是一些实实在在的东西。下文中我用团队成员的英文名来代替中文名,分别是:\textcolor{colorname}{Barry}——李彦迪;\textcolor{colorname}{Shawn}——陈硕烁;\textcolor{colorname}{Jake}——赵俊博

\section{用技术打造不同的申请模式}
\subsection{Delta-T-Obsever——by Shawn}
Obsever,“观察者”。这是一个很优秀的平台,它实时关注美国大学的教授动态:例如谁谁谁又要招PhD了、哪个教授又有RA的机会放出来了之类。同时,还是上次提到过的问题,申请是面向Program的。因此很多时候新Program的推出往往意味着小的竞争压力(例如今年约翰霍普金斯大学(JHU)新推出Robotics的MS项目)。综上这些情况Obsever可以通通告诉你。并且Obsever是一个实时发布的平台,可以在第一时间发布你想要的信息——新的机会、新的项目。Obsever由Shawn负责,目前已经开始筹备,\\ Code请见:\href{https://github.com/0x000fff/Delta-T-Observer}{\textcolor{colorname}{Code @ github}}
\subsection{Delta-T-Prophet——by Jake}
Prophet,“先知”。"We are living in the era of Big Data!",大家每个人都在说“大数据”、“大数据”……那有没有可能用大数据的视角来看申请?各个论坛上的申请结果数据多如牛毛,如果把所有的数据都收集在一起,然后面对每一所学校、使用相应的数据做拟合从而预测未来的申请结果,会怎么样?这就是Prophet要做的事情——使用科学模型+数据来说话,我们相信这样的方式可以给申请带来“量化”的质变,而不再是盲目得比对。对于申请,数据分析和机器学习可以帮助你!做到后期我们会考虑将Prophet升级成“推荐系统”,智能化、理性化得为申请者推荐学校。Prophet主要由Jake负责,目前也已经开始筹备,\\Code请见:\href{https://github.com/zhaojunbo/Delta-T-Prophet}{\textcolor{colorname}{Code @ github}}

\section{大家都关心的收费问题}
昨晚,Delta-T开会开了3个小时,讨论这个问题。最后定下来的方案如下,方案中包含了一些粗略的工作设计和相应的收费情况:
我们的收费标准是打包制定的,也就是——选校、文书、其它服务分别按照具体的工作量定价,当然也可以单独购买某一个包的服务。
\begin{itemize}


\item{选校——根据客户的想法,兼顾“面向排名选校”和“面向教授”两类选校模式——前者选校主要关注排名和名气、后者则主要跟随大牛级的教授而选。\\收费标准:4950 RMB / 10所;6950 RMB / 15所 \\
	  选校这部分工作,Delta-T会对每一所学校都会出示具体的一份报告,报告中囊括:学校基本信息、综排和专排、IEEE/ACM Fellow个数、项目基本信息(多少学分毕业、多少课、时间长短、thesis、费用问题)、院系和周边的企业合作情况、地缘优势、招生情况(optional)、针对申请方向知名的教授(近期工作、实验室情况、实验室华裔比例、手下有无PhD毕业而位置空缺、有无套磁可能)。如果你去看看留学中介如何做选校工作,就能看到和Delta-T的差距了。}
\item{PS/CV/推荐信,即文书材料写作指导——Delta-T的重点工作定位在指导,不代写;重在内容,负责框架和写后修改(修改时间有限制:第一所学校申请截止日期前两周之后不予修改) \\ 收费标准: 4950 RMB。PS若方向不一致:根据具体方向的相关度和文书材料的重叠度,考虑加价500-1500 RMB不等(相关度如何定义,可提前咨询商议)}
\item{技术性培训: \\ 学术性写作1套课程:{\indent	\indent} 980 RMB		\\		面试套磁和教授礼仪课程:{\indent \indent} 280 RMB}
\item{免费送签证培训、签订学校、入学前相关事宜(体检、疫苗)、多校择Offer等服务。}
\end{itemize}
\textcolor{colorname2}{注:上述工作和收费标准主要针对大三同学;对于大二的同学,你们可以考虑暂不签订选校包和文书包的服务,先参加Delta-T的培训,积累起来paper和干货,到申请的时候再做决定是否要签订选校和文书包服务!应该说对于你们来说还是有一些时间可以踏踏实实得学点东西、做点事情,其实你们这个时间是绝对的黄金时间,Delta-T可以帮你们来规划你们之后的工作!}

\section{Next time!}
之后的这段时间,Delta-T的主要工作为:\begin{itemize}
\item{Obsever和Prophet}
\item{完善合同等法务问题}
\item{制作一份选校的报告样本,初定以计算机科学方向久负盛名的Carnegie Mellon大学为对象。}
\item{继续欢迎大家骚扰。}
\end{itemize}

\section{最后说几句}
上次发完日志之后,很多人在下面点赞,我很高兴;但是也有不少童鞋喷我说写软文、做广告。首先,我非常理解你们这种想法,但是我想在这里澄清几件事情:
\begin{itemize}
\item{喷我的童鞋你们大多不认识我们~所以不了解我们的为人。Delta-T的成员,都是非常负责任、讲信用的人。团队整个不是money-driven,更多的是情怀-driven,我们想真正得做点事情,帮助更多的学弟学妹们申请,才走出的这一步。}
\item{费用问题今天公布了,不知道同学们看着会有什么反映。我可以这么算一笔账,如果你接受了Delta-T的全部服务,在选10所学校的情况下,总费用大概12000 RMB不到。但在同等条件下,你去找留学中介的话花费会在3倍以上。同时我想说Delta-T的工作质量肯定比全世界任何一家留学中介做得都要高。这一点,我既然敢公开得这么说,就说明我有绝对的信心实现它。}
\item{很多人会抓着Delta-T不全包写文书、只管指导和修改这件事不放,表示我们不如中介。两个观点,第一,中介在你的文书上花的时间是多少?如果你了解行情的话,现在基本上每个中介老师手上平均的学生数目在15-45不等,稍微一算就知道他们在你的文书身上能花多少时间了。第二,中介肯定是会帮你写文书,但是据我和我身边的朋友来看,中介写出来的文书基本上百分之60-100都会被修改乃至重新写过,而这些工作谁做的呢?All by yourself!我还是那句话,申请CS的同学你们不可能让一个连Hello World都不会写的人帮你写文书吧?我们的观点是,申请的过程、写文书材料的过程是对自己的一个磨砺——至少我认为我的英文在申请后比申请前强了很多。做申请的时间不短,这段时间我不希望童鞋们停止前进的步伐,所以说这其实这是个很好的增加自己水平的机会。不过你也别害怕,Delta-T会一直帮助你。}
\item{最后的几句“废话”。Delta-T来自武汉大学,今年我们三人刚刚毕业。不避讳得说,我们都能看到武大有问题,很多的地方或许已经和清华、北大拉开了不少的差距。最简单的,出国这一块我们在申请的时候都感到了无力。清华北大、中科大等学校,每年出国人才济济,这些校友为自己的母校留下了良好的口碑(申请的时候你就会发现这起到了多少作用);但是武大,EE/CS这一块,出国的学长学姐们数量相比他们少了太多,因此才会在市面上广泛流传所谓的“老外眼里清华北大中科大一个层次,其它学校都一样”的说法。\\值此毕业之际,我不会再想说母校哪里哪里不好,而是饱含热泪得站在“国立武汉大学”的牌坊前,深情得望着那里的一花一草一木。我想说,身为一个本科的毕业生,我们团队一直在想我们能为母校做点什么?Delta-T不是个商业组织,也不是留学中介,建立的目的无非就是想帮助学弟学妹们做好申请,希望他们今年申请不会像我们去年那样感到无力和艰辛。\textcolor{colorname}{其实我希望看到的不仅仅是找我们的童鞋可以得到到牛掰的学校或者牛掰的教授青睐,更希望看到的是一种“传承”——如果今年愿意找我们的学弟学妹们在申请结束之后也去帮助你们之后的学弟学妹,我就可以含笑九泉了。我相信武大会因为我们和之后的你们的努力变得更好。}\\
\textcolor{colorname2}{当然啦,其它学校的童鞋,也非常非常欢迎你们来联系Delta-T。申请乃切肤之痛,人皆有之,Delta-T不会让你失望。}}
\end{itemize}
	
\section{联系方式更新}
\begin{itemize}
\item{如果你对网络、嵌入式、体系结构感兴趣,请联系:陈硕烁\\
在线简历:0x000fff.github.io\\
Skype ID: shawnchen.ss\\
Email: sschenshawn@gmail.com\\
QQ: 2339815524}
\item{如果你对计算机视觉、机器学习、信号处理感兴趣,请联系:赵俊博\\
在线简历1:zhaojunbo.github.io\\
在线简历2:\href{https://www.dropbox.com/s/layknk0qbke541c/res9a_JunboZhao.pdf}{\textcolor{colorname}{On Dropbox}}\\
Skype ID: zhaojunbo1992\\
Email: zhaojunbo1992chasing@gmail.com\\
QQ: 309390493}
\item{其他的方向或咨询,请联系:李彦迪\\
Email: delta-t@foxmail.com\\
QQ: 1062339067}
\end{itemize}
\\
\vspace{20pt}
{\large\bf Delta-T欢迎大家骚扰~}

\end{document}





