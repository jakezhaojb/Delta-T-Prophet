\documentclass{article}
\usepackage{xeCJK}
\usepackage{color}
\definecolor{colorname}{rgb}{0,0.12,0.60}
\definecolor{colorname2}{rgb}{0.60,0.12,0}
\usepackage{hyperref}
\setCJKmainfont{AR PL UKai CN}
\begin{document}
\moveleft.5\hoffset\centerline{\huge\bf 完整申请流程及Delta-T工作初步安排}
\vspace{5pt}
\moveleft.5\hoffset\centerline{\large\bf Application pipeline and Delta-T's schedule}
\vspace{15pt}
\moveleft.5\hoffset\rightline{\bf By Delta-T}

\vspace{25pt}
\moveleft.5\hoffset\centerline{\textcolor{colorname}{\large\bf Tools used by Delta-T}}
\begin{itemize}
\item{通讯工具:{\bf Skype} \\ Delta-T使用的标准通讯工具是Skype,通常情况下mentor会和自己的客户根据申请的日程安排约好时间,保持至少每周一次的Skype语音通讯——主要是询问进度进行情况、制定下周的工作安排、例行的讲解和答疑解惑。}
\item{大文件共享工具:\textbf{Baidu-cloud} \\
即时信息:{\bf Skype} \\
非紧急信息:{\bf Email} \\ 一般情况下,非紧急事件发Email给自己的mentor询问,mentor看到邮件之后回复,回复的迟滞时间不会超过1天;较为紧急的事件可以使用Skype的留言功能留言解决;对于文件的传输和共享问题,小文件可以通过邮件传输,大文件(超过邮件附件上传限制的)可以使用百度云盘私密共享。 }
\item{Delta-T常规工作时间:北京时间每周三、周五上午10-11点,周日晚上10-11点/ 10-11pm Tuesday \& Thursday, 10-11am Sunday EST \\ Delta-T设立常规工作时间,这段时间会有专人在Skype线上帮助大家解答各种问题。 }
\end{itemize}

\vspace{25pt}
\moveleft.5\hoffset\centerline{\textcolor{colorname}{\large\bf 16 things to do in whole application}}

\section{打印、寄送三年成绩单}
首先更正一个误区,申请的时候GPA是self-report的,也就是教务部出示的标准样本中英文成绩单是没有GPA的(至少现在是这样)!Delta-T也并不建议同学们使用武大算法计算的GPA去申请,平均分有时候更有说服力。下面是具体攻略:\\
\textcolor{colorname2}{没有需要删除的公选课},直接去教务部一楼大厅机器上打印英文版成绩单,用校园卡就能打印。具体需要注意的问题是:打印及格重修成绩的程序系统可能会被教务部修改,目前的设置是:“审核前成绩”是及格重修覆盖之后的成绩单,“审核后成绩”是及格重修未被覆盖的成绩单,请同学们打印时务必要注意这一点,之后的程序可能还会被修改,请检查之后再打。收费情况与找老师打印相同。 \\
\textcolor{colorname2}{若需要删除公选},先去院办找吴霞打印中文版成绩单并盖学院公章,然后到教务部一楼大厅窗口处咨询,说明需要删课。工作人员会告诉你把及格重修的课标出来,并且在满足公选学分的情况下划掉要删的公选。然后排队领取一个档案袋,领到后填档案袋封面的信息并去财务室交钱,财务部的人会根据你档案袋上写的需要多少份成绩单来计算收费,交完钱后拿着收据再回到窗口交中文成绩单。交完后等周五去取即可。注意如果找工作人员删除专选成绩,成绩单的下方会出现一行字:"部分课程被隐藏"。 \\
\textcolor{colorname2}{关于四、六级}:默认情况下,四、六级的成绩都会出现在成绩单上;四级学分为0,作为必修课(required)出现、六级学分为4分,作为选修课出现(elective)。四六级在成绩单上成绩是按照百分制核算而来,六级若成绩不高有两种处理方法,(1). 若在教务部工作人员处删除,则会出现那行字。(2).可以在教务部1楼右拐第一个办公室的工作人员处彻底抹除六级成绩,之后六级成绩将不会在成绩系统中出现。如果没有其它要删除公选课的同学可以直接去机器上解决了。\\
\textcolor{colorname2}{关于打印份数}:建议打印成绩单的数量比申请学校的数目稍微多几份,原因是可能有些学校需要寄送2份成绩单。\\
\textcolor{colorname2}{关于在读证明}:建议同样打印在读证明,份数与成绩单相同,寄送成绩单时将在读证明与成绩单一同寄送。且不要把在读证明也封进武大信封。在读证明在机器上使用校园卡直接打印。\\
\textcolor{colorname2}{成绩单翻译受理时间}:周一下午,周二上午\\
\textcolor{colorname2}{领取时间}:周五全天;机器打印时间不限\\
\textcolor{colorname2}{成绩单收费}:50 RMB 翻译费 +  10 RMB / 份\\
\textcolor{colorname2}{在读证明收费}:5 RMB / 份\\
\textcolor{colorname2}{关于寄送成绩单}:首先需要根据各个学校的要求来准备,例如有些学校不需要寄送纸制成绩单,只需要扫描版的(unofficial transcript);有的学校需要一份,有的学校需要两份等等。具体的操作流程是将纸制的成绩单放入武汉大学信封并在加盖公章的封口处使用胶条封死信封,这样算作一份officail transcript。之后,配上相应份数的在读证明,可以通过联系DHL、UPS或者Fedex寄送到学校给出的接收地址。一般情况下,三天之内可以送到,费用在每份200 - 300元左右(准备好银子)\\
\textcolor{colorname2}{Tips}:
(1)如果覆盖了及格重修,该门课程是出现在及格重修的那一学期中的。如果删了公选,在成绩单底部会出现一行字:该份成绩单隐藏了培养计划外的无关课程;\\
(2)英文版成绩单实际上是中英文对照的,上面没有任何GPA,加权平均分等,只有各门课的分数;\\
(3)每份成绩单会配给一个武大信封,寄送成绩单时把成绩单装入之后自己封好信封,信封上有已经盖好的公章,代表此份成绩单由武大教务部开具并密封。一个密封好的成绩单信封就算是一个official transcript了。切记不要在信封或成绩单上写任何东西。\\
\\
\indent\textcolor{colorname2}{\bf Delta-T这里的主要工作有:}
\begin{itemize}
\item{指导、帮助整个过程,虽然写得很清楚,但是真的可能中途出BUGs......}
\end{itemize}
\textbf{Recommended deadline: Prior to application deadline}

\section{获得GRE \& TOEFL(IELTS)纸质成绩单}
G和T的纸质成绩单在申请时是不需要的,但是作为唯一的正式成绩单可能在签证或者办理入学手续时被要求出示。\\
\indent 获得G\&T成绩单的前提是考完G\&T,并保证成绩到申请之后有效。ETS会将成绩单送到考场,再由考场根据考G\&T时填写的mail address将纸质成绩单通过EMS寄送给你。\\
\textbf{Recommended deadline: October 1st}

\section{办理护照}
可能有的申请者一直都没有自己的护照,这是在办理签证时需要的,建议提前准备。可以去学校所在地的公安局出入境管理处或者户口所在地公安局出入境管理处办理,需要X个工作日(需要时间因地区差异不定)。具体办理流程网上搜索即可。\\
\textbf{Recommended deadline: September 1st}

\section{办理一张双币信用卡}
在申请时付费可能需要支付美元,此时需要有一张能够支付美元的双币信用卡。各位同学可能还不能办理信用卡,一般是让父母或朋友办理,使用他们的子卡。\\
\textbf{Recommended deadline: September 1st}

\section{注册一个正式邮箱}
在申请过程中有许多地方需要填写自己的联系方式,一般为电子邮件。这将会是学校等各个机构或个人联系你的首选方式。因此最好注册一个比较正式的邮箱,建议使用学校邮箱或者gmail,yahoo mail,hotmail等商业邮箱。\\
\indent 强烈不建议使用qq邮箱,因为据许多人反应qq邮箱发送的邮件经常被当作spam而遭到屏蔽。\\
\indent 同时建议给自己的邮件地址设置一个alias,比如whu邮箱最好不要用学号,而是改成自己的名字如\textcolor{colorname}{san.zhang@whu.edu.cn}(注意外国人的名字习惯是名在前姓在后)。此外也要设置一个落款,这样在给教授或学校发送邮件时能够使对方比较容易的知道你是谁。落款一般为:
\begin{description}
{\small Regards,\\
San Zhang,\\
School of XXX,\\
Wuhan University}
\end{description}\\
\textbf{Recommended deadline: Today!}

\section{选校并注册申请系统}
到这一步时你应该基本已经确定了要申哪些学校,一旦确定后就先去各个学校的申请系统里注册一个帐号,这样开始申请时就能直接登陆了。\\
\\
\indent\textcolor{colorname2}{\bf Delta-T这里的主要工作有:}
\begin{itemize}
\item{出示每个候选学校的详尽报告,内容主要包含:学校基本信息、综排和专排、IEEE/ACM Fellow个数、项目基本信息(多少学分毕业、多少课、时间长短、thesis、费用问题)、院系和周边的企业合作情况、地缘优势、招生情况(optional)、针对申请方向知名的教授(近期工作、实验室情况、实验室华裔比例、手下有无PhD毕业而位置空缺、有无套磁可能);}
\item{根据{\bf Observer}和{\bf Prophet}智能化推荐选校;}
\item{指导权衡的方法——"究竟怎么看学校,排名能决定一切吗?"}
\end{itemize}
\textbf{Recommended deadline: November 15th}

\section{推荐信}
首先应确定推荐信的目标老师,一般来说大部分申请学校要求3个老师做推荐(个别学校可能是2个,但是第三个老师通常也可以提交)。3个老师的安排应该尽可能合理——不同的方面来展现你的特点。建议来说,不要3个老师都找自己院的老师。这里给出一个标准配置:可以1个老师找分数比较高课程的任课教师展现课堂实力;第2个老师找指导过科研的老师展现科研能力;第三个老师建议找实习或者工作的指导人,展现工程或协作能力。Delta-会根据个人的情况推荐推荐信老师的配置。确定推荐的老师之后,应该到老师办公室提出推荐的请求,然后大部分老师会把写推荐信的工作交给学生自己;少部分老师会选择自己负责推荐信的写作。之后根据老师的要求,站在老师的视角上撰写推荐信。\\
\\
\indent\textcolor{colorname2}{\bf Delta-T这里的主要工作有:}
\begin{itemize}
\item{会指导和帮助修改——老师的推荐信有很多需要注意的地方,例如应该写出高度和广度,不要拘泥于细节等等。}
\item{Delta-T团队成员认识电信院各个实验室的老师,如果可以的话会帮助在这方面牵线搭桥。}
\end{itemize}
\textbf{Recommended deadline: December 1st, but start finding professors from now!}

\section{PS / CV}
Personal Statement:展现专业实力的最好机会,PS里面应该展现出自己和申请专业相关的经历或者想法。经历方面,可以写一些相关的课程设计工作、科研工作或者相关的实习经历,这里面应该既包含综述又有具体细节,细节更具有说服力。想法方面,可以将自己对于专业的看法、或者看paper之后的感想展现在上面,此外还可以在结尾表达对该学校某个教授的兴趣。需要注意的是,写PS和写GRE的AW一样,要有清晰的逻辑结构,要有表达的重点和核心。最好是每段话都能够展现你作为申请者的一些优点或特质。同时作为工科申请者,PS不需要过分文采飞扬,不必过于纠结单词与语法,能够清晰的表达自己就够了。\\
\indent Curriculum Vitae: 非常重要的申请材料——简单、直接展现自己经历和能力的材料。一份像样的简历不光是留学申请的敲门砖,求职、找实习,甚至相亲,简历都是至关重要的!\\
\\
\indent\textcolor{colorname2}{\bf Delta-T这里的工作主要是:}
\begin{itemize}
\item{帮助指导PS的写法;}
\item{Delta-T成员具备一定专业实力和素养,可以帮忙在PS中加入一些专业性的内容,一句话说,“我们知道或者能想到的,就是你PS里的!让我们用大家的智慧,推起干货抓住美国人的眼~”。}
\item{Delta-T了解一些学校或者教授,可以在PS的最后一段,就是对于针对每个学校专门撰写的段落中,争取写出切中肯綮的东西。}
\item{Delta-T指导撰写和修改,并且建议使用TeX制作专业的CV!}
\end{itemize}
\textbf{Recommended deadline: Prior to application deadlines}

\section{寄送T(I) \& G成绩单}
托福成绩单通过NEEA系统直接寄送,支持网银付费;GRE成绩需要通过ETS官网寄送,使用信用卡付费;雅思的成绩由雅思官方网站寄送。\\
\textbf{Recommended deadline: One month before application deadlines}


\section{网申}
填写网申系统,在里面自己报告GPA、G、T成绩,并且上传PS、CV等附件。并且,在网申系统中填写推荐教授的邮箱,系统会自动发送邮件到教授的邮箱(见下一章)。填写完毕且没有出错之后提交,付款,申请结束。最后的界面一般是一个状态跟踪网页,可以看到T,G,成绩单等是否被收到(具体请看第11章)。\\
\textbf{Recommended deadline: Prior to application deadlines}

\section{在线提交推荐信}
理论上来说,这个网上提交推荐信的工作应该是由教授自己完成的。但是考虑到具体情况——大部分的教授都不是自己来写推荐信,这里分两种情况考虑。第一种,就是一般情况下不是教授来写推荐信,在学生的要求下,教授将网申系统自动给他们发的邮件转发给学生,里面会有一个链接。学生收到邮件后直接点开链接,便可以看到提交推荐信的系统,除了上传推荐信的附件(先打印 -> 教授签字 -> 扫描存成pdf)之外,还需要填写一些评价或者评分内容。第二种,如果教授自己完成推荐信的撰写,学生便要求教授尽快完成提交和评价工作。\\
\indent 注意,有些学校也接受纸制推荐信。 \\  
\textbf{Recommended deadline: Prior to application deadlines}

\section{查看申请状态}
申请提交之后就进入了漫长的等待offer阶段。美国的学校一半在周一到周五会不定期更新申请系统的状态,周末及美国法定节假日不上班。因此你可以计算好时差定时查看申请系统的状态更新。当然也有些学校并不通过申请系统,而是通过邮件告知你进展。\\
\indent 需要留意的几个比较重要的时间节点是,GT及成绩单送达学校,申请已进入院系审核,开始发放录取通知。\\
\indent 通过网申系统可以看到GT分数及成绩单是否已经被收到,如果均已收到且推荐信也全部提交,则会显示申请complete,此后会受到申请系统或者学校小秘的邮件通知,并告知已经开始审理你的申请。\\
\indent 有些学校会另外通知你你的申请材料已由研究生院转至院系开始由录取委员会审核。\\
\indent 最后关注相关留学论坛,看看其他申请者报告的消息,比如某一天某所学校开始大量发放录取等等。\\
\\
\indent\textcolor{colorname2}{\bf Delta-T这里的工作主要是:}
\begin{itemize}
\item{帮助指导如何与对方小蜜Email沟通;}
\item{必要时,帮助给对方学校的小蜜做Skype电话沟通}
\end{itemize}
\textbf{Recommended deadline: Prior to application deadlines}


\section{准备面试}
如果你申请了某些学校的TA/RA或者奖学金或者是直接申请的phd,那么很有可能这所学校的教授会联系你要进行面试。面试一般是通过skype进行,你需要提前了解这个教授的研究方向,近期研究成果,他可能会问的问题等等。


\section{确定学校}
如果申请比较成功,那么你会拿到多所心仪的学校的录取。这个时候你需要好好考虑,包括专业,地理,环境,气候,机会等各方面因素。另外由于各个学校给的考虑时间不同,可能会导致你不得不在更早的一个时间之前做出决定。\\
\indent 至于如何选择,Delta-T只能从客观的角度给出参考性意见,比如A学校的专业看起来比B学校更好;A学校所在地理位置更好,有利于未来就业;B学校学费便宜且第二年更有希望获得奖学金等。但是毕竟这是你的人生,你的未来,一切决定权还是在你,Delta-T不会做任何干涉。

\section{I-20 \& VISA}
在你确定要接受某一所学校并向学校确认后,开始进入录取流程。首先,学校负责招生事务的官员或办公室会要求你提供你的护照ID页扫描件,亲笔签名的录取信扫描件等。然后学校随后会将一个名为I-20的表格通过快递寄给你。你需要在I-20上签字,并以此作为申请签证的有效证明材料。\\
\indent 接下去到美国大使馆官网上查看申请F-1签证需要的准备材料和流程。一般有I-20,录取信,护照,DS-160,SEVIS付费确认信,面签预约确认信,个人简历,导师简历,财产证明,父母工作证明,房产证,存折等。并非全部必须,但是最好准备齐全以备需要。\\
\indent 如果面签顺利则在3个工作日后可以拿到签证,若被Administrative Processing则要等2-4周拿到签证。一半没有被当场拒签的情况。但切记签证材料和信息不要造假。

\section{出发前准备}
在学校已经确定的情况下,需要做三件主要的事:租房,订机票,体检。另外还再锻炼下交流能力,主要是口语和听力。\\
\indent 租房主要选离学校近,安全,环境好且价格合适的房子。一般来说你会和你的同学或者朋友合租。找房子一般在craigslist还有各个学校的housing页面上查看,也可以问学长学姐。\\
\indent 订机票越早越便宜,在能确定出发日期的情况下尽可能早的订机票可以买到便宜的机票。需要注意的是第一次去可能行李比较多,而一般航空公司对于行李是有一些限制的。比如限制免费托运一件行李,多出的需要付费;对带上飞机的行李有尺寸和重量的限制等。另外也需要注意到达时间,如果到达时间过早或过晚可能没有交通方式能带你去学校。\\
\indent 体检分为体检和疫苗两部分,每个学校在入学前都会要求打相应疫苗,可以在当地出入境检验检疫部门完成。


\end{document}





